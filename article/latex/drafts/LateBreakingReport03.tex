% This is "sig-alternate.tex" V2.1 April 2013
% This file should be compiled with V2.5 of "sig-alternate.cls" May 2012
%
% This example file demonstrates the use of the 'sig-alternate.cls'
% V2.5 LaTeX2e document class file. It is for those submitting
% articles to ACM Conference Proceedings WHO DO NOT WISH TO
% STRICTLY ADHERE TO THE SIGS (PUBS-BOARD-ENDORSED) STYLE.
% The 'sig-alternate.cls' file will produce a similar-looking,
% albeit, 'tighter' paper resulting in, invariably, fewer pages.
%
% ----------------------------------------------------------------------------------------------------------------
% This .tex file (and associated .cls V2.5) produces:
%       1) The Permission Statement
%       2) The Conference (location) Info information
%       3) The Copyright Line with ACM data
%       4) NO page numbers
%
% as against the acm_proc_article-sp.cls file which
% DOES NOT produce 1) thru' 3) above.
%
% Using 'sig-alternate.cls' you have control, however, from within
% the source .tex file, over both the CopyrightYear
% (defaulted to 200X) and the ACM Copyright Data
% (defaulted to X-XXXXX-XX-X/XX/XX).
% e.g.
% \CopyrightYear{2007} will cause 2007 to appear in the copyright line.
% \crdata{0-12345-67-8/90/12} will cause 0-12345-67-8/90/12 to appear in the copyright line.
%
% ---------------------------------------------------------------------------------------------------------------
% This .tex source is an example which *does* use
% the .bib file (from which the .bbl file % is produced).
% REMEMBER HOWEVER: After having produced the .bbl file,
% and prior to final submission, you *NEED* to 'insert'
% your .bbl file into your source .tex file so as to provide
% ONE 'self-contained' source file.
%
% ================= IF YOU HAVE QUESTIONS =======================
% Questions regarding the SIGS styles, SIGS policies and
% procedures, Conferences etc. should be sent to
% Adrienne Griscti (griscti@acm.org)
%
% Technical questions _only_ to
% Gerald Murray (murray@hq.acm.org)
% ===============================================================
%
% For tracking purposes - this is V2.0 - May 2012

\documentclass{sig-alternate-05-2015}


\begin{document}

% Copyright
\setcopyright{acmcopyright}
%\setcopyright{acmlicensed}
%\setcopyright{rightsretained}
%\setcopyright{usgov}
%\setcopyright{usgovmixed}
%\setcopyright{cagov}
%\setcopyright{cagovmixed}


% DOI
\doi{10.475/123_4}

% ISBN
\isbn{123-4567-24-567/08/06}

%Conference
\conferenceinfo{HRI '17}{March 6--9, 2017, Vienna Austria}

\acmPrice{\$15.00}

%
% --- Author Metadata here ---
\conferenceinfo{HRI}{'2017 Vienna, Austria}
%\CopyrightYear{2007} % Allows default copyright year (20XX) to be over-ridden - IF NEED BE.
%\crdata{0-12345-67-8/90/01}  % Allows default copyright data (0-89791-88-6/97/05) to be over-ridden - IF NEED BE.
% --- End of Author Metadata ---

% \title{Towards the Understanding of Human Movement Variability with NAO as an Dance Instructor}
% \title{Towards The Classification of Human Movement Variability in Dance Activities Using NAO as an Instructor}
% \title{Towards The Classification of Movement Variability For Simple Arm Activities Using NAO as an Instructor}
% \title{Can humanoids robots teach humans how to move?}
% \title{Can a robot teach you how well you move?}
% \title{Can NAO be used to improve the quality of movement?}
% \title{Can NAO be used to analyse the quality of simple movement?}
% \title{DRAFT: Can NAO be used as a model of simple movement to analyse the variability 
% of movement?}
\title{DRAFT: Can a humanoid-robot 
be used as a model of simple movement to analyse the movement variability 
accross participants and across movements?}


% \subtitle{[Extended Abstract]
% \titlenote{A full version of this paper is available as
% \textit{Author's Guide to Preparing ACM SIG Proceedings Using
% \LaTeX$2_\epsilon$\ and BibTeX} at
% \texttt{www.acm.org/eaddress.htm}}}
%
% You need the command \numberofauthors to handle the 'placement
% and alignment' of the authors beneath the title.
%
% For aesthetic reasons, we recommend 'three authors at a time'
% i.e. three 'name/affiliation blocks' be placed beneath the title.
%
% NOTE: You are NOT restricted in how many 'rows' of
% "name/affiliations" may appear. We just ask that you restrict
% the number of 'columns' to three.
%
% Because of the available 'opening page real-estate'
% we ask you to refrain from putting more than six authors
% (two rows with three columns) beneath the article title.
% More than six makes the first-page appear very cluttered indeed.
%
% Use the \alignauthor commands to handle the names
% and affiliations for an 'aesthetic maximum' of six authors.
% Add names, affiliations, addresses for
% the seventh etc. author(s) as the argument for the
% \additionalauthors command.
% These 'additional authors' will be output/set for you
% without further effort on your part as the last section in
% the body of your article BEFORE References or any Appendices.

\numberofauthors{2} %  in this sample file, there are a *total*
% of EIGHT authors. SIX appear on the 'first-page' (for formatting
% reasons) and the remaining two appear in the \additionalauthors section.
%
\author{
% You can go ahead and credit any number of authors here,
% e.g. one 'row of three' or two rows (consisting of one row of three
% and a second row of one, two or three).
%
% The command \alignauthor (no curly braces needed) should
% precede each author name, affiliation/snail-mail address and
% e-mail address. Additionally, tag each line of
% affiliation/address with \affaddr, and tag the
% e-mail address with \email.
%
% 1st. author
\alignauthor
XXX XXX\titlenote{xxxx xxxxx xxxx xxxxx xxxx xxxxx xxxx xxxxx}\\
       \affaddr{University of XXX}\\
       \affaddr{XXX,XXX}\\
       \email{xxx@xxx.xxx}
% 2nd. author
\alignauthor
XXX XXX\titlenote{xxxx xxxxx xxxx xxxxx xxxx xxxxx xxxx xxxxx}\\
        \affaddr{University of XXX}\\
       \affaddr{XXX,XXX}\\
       \email{xxx@xxx.xxx}
% % 3rd. author
% \alignauthor 
% XXX XXX\titlenote{xxxx xxxxx xxxx xxxxx xxxx xxxxx xxxx xxxxx}\\
%        \affaddr{University of XXX}\\
%        \affaddr{XXX,XXX}\\
%        \email{xxx@xxx.xxx}
}
% There's nothing stopping you putting the seventh, eighth, etc.
% author on the opening page (as the 'third row') but we ask,
% for aesthetic reasons that you place these 'additional authors'
% in the \additional authors block, viz.
% \additionalauthors{Additional authors: John Smith (The Th{\o}rv{\"a}ld Group,
% email: {\texttt{jsmith@affiliation.org}}) and Julius P.~Kumquat
% (The Kumquat Consortium, email: {\texttt{jpkumquat@consortium.net}}).}
% \date{30 July 1999}
% Just remember to make sure that the TOTAL number of authors
% is the number that will appear on the first page PLUS the
% number that will appear in the \additionalauthors section.

\maketitle
\begin{abstract}
This paper provides ... 
% a sample of a \LaTeX\ document which conforms,
% somewhat loosely, to the formatting guidelines for
% ACM SIG Proceedings. It is an {\em alternate} style which produces
% a {\em tighter-looking} paper and was designed in response to
% concerns expressed, by authors, over page-budgets.
% It complements the document \textit{Author's (Alternate) Guide to
% Preparing ACM SIG Proceedings Using \LaTeX$2_\epsilon$\ and Bib\TeX}.
% This source file has been written with the intention of being
% compiled under \LaTeX$2_\epsilon$\ and BibTeX.
\end{abstract}


%
% The code below should be generated by the tool at
% http://dl.acm.org/ccs.cfm
% Please copy and paste the code instead of the example below. 
%
\begin{CCSXML}
<ccs2012>
<concept>
<concept_id>10010520.10010553.10010554.10010558</concept_id>
<concept_desc>Computer systems organization~External interfaces for robotics</concept_desc>
<concept_significance>500</concept_significance>
</concept>
</ccs2012>
\end{CCSXML}

\ccsdesc[500]{Computer systems organization~External interfaces for robotics}


%
% End generated code
%

%
%  Use this command to print the description
%
\printccsdesc

% We no longer use \terms command
%\terms{Theory}

% \keywords{ACM proceedings; \LaTeX; text tagging}
\keywords{Humanoid dance robot, movement variability, Wearable sensors, Inertial sensors}

\section{Introduction}
The use of humanoid-robots has been increasing for training and demonstration of dance movements.
For instance, Fung \textit{et al.} made movement comparison of 
two low-cost humanoid robots for implementation and demonstration of 
simple Thai dance activities  \cite{Fung2008}. 
Xia \textit{et al.} implemented automatic humanoid-robot dancing with NAO
which is driven by the beats and emotions of the music \cite{Xia2012}.
Similarly, NAO acted as a tutor to guide children to teach dance with the 
\textit{concept-based learning} in which the robot guides the child 
thorough different stages of dance \cite{Ros2013,Ros2014}.
% Additionally, Ros \textit{et al}.  explored three different learning models to design a robot dance tutor:
% i) Sequence-Based Model; (ii) Concept-Based Model; and 
% (iii) Relational-Based Model  \cite{Ros2013}. 
In the same vein of dancing with robots, \textit{Keepon}, a non-humanoid-robot, 
showed a positive effect on the children's rhythmic behavior.
% However, many open pathways of human-robot interactions remained to be explored. 
For example,
some children tried to make the robot synchronise its movements to the music,
some other children exaggeratedly danced to the music when the robot was not 
synchronised; while other children imitate robot's movements to the point
of ignoring the musical rhythm, and one important behavior was that the
synchrony of the children with the music was more prevalent than the robot movement \cite{Michalowski2007}.
Tsuchida \textit{et al.} explored the dancing feeling with an actual dancer in four scenarios 
(dancing with a dancer, dancing alone, dancing with a self-propelled robot
and dancing with a projected video) of which they managed to make
people feel like they were dancing with a dancer
when dancing with a self-propelled robot and dancing with a projected video \cite{Tsuchida2013}.
% However the system is lacking of appropriate trajectory information of the 
% participants .
However, none of the previous works have been analysed the movement of the participants 
whom interact with the robot. For this report, we are therefore asking if the use 
of NAO can be useful to analyse the variability across movements 
and across participants performing simple movements.

The report is divide in three sections: 2. 



% For our exploration purposes of movement variability, we adopt the Sequence-Base Model
% in which the movement sequences are created by the teacher and the pupils
% reproduce it in the same way.
% It is important to note that the teaching methodology is the same but the movements performed
% by users were completely different between sessions.
% However, there is no analysis about the quantification of the quality of the
% improvement of the movements as children performed in each of the three sessions 
% but only a score for accommodation that is a ratio between the number of times when robot shows and asks
% for a motion of the user \cite{Ros2014}.





% 
% 
% 
% Chen et al. proposed a tennis swing training using accelerometer sensors which classifies 
% four categories of motion 
% (correct forehand swings, only swing wrist, arm unstretchable and racket is not vertical) 
% with a ID3 inductive learning algorithm \cite{Chen2010}.




% FUTURE WORK:
% One interesting factor when designing a dance robot tutor is preventing the robot to 
% behave as a static machine, to do this the robot includes head random movements
% blinking eyes, spatial orientation and name reference.
% * Methodology for creative dance



\section{METHODS}

\subsection{Research Questions}

Can the use of a humanoid Robot help us to understand the measurament of movement variability
accross participants and accross movevements?
How the variability of a movement can be related to the dexterity of the performance
of such movement?


\subsection{Hyphotheses}
 We belive that the outcome that this preliminary experiment 
  can help us to understand the variability of simple movements
  which for future outcomes the variablity of the movements
  can be linked with the leve dexterity of users and more 
  importantly provide feedback to move better.
\subsection{Participants}
Twelve participants.
\subsection{Procedure}
Participants were asked to follow two simple movements which were performed by a NAO robot.
Two inertial sensors were attached to the right hand of the robot and two intertial sensors were attached
to the right hand of the participant. We use muse inertial sensors [ADD REF]
which provide tri-axial data for the accelerometer, gyroscope and magnetometer as well
as quaternions.



\section{PRELIMINARY RESULTS AND OUTGOING DATA ANALYSIS}



Please see Figure ~\ref{fig:main} 
% affiliation/
\begin{figure*}
\centering
\includegraphics[width=1.0\textwidth]{fig00}
\caption{Ten repetitions of vertical and  horizontal movement were performed by twelve participants 
following a humanoid robot.
Error bars present the mean values for ACC and GYR sensors for participants and the robot
of which the sensors 01 and 02 were attached to participants
 and the sensors 03 and 04 were attached to the robot. 
(x) Time-series for s01-s04 for less and more sync horizontal movement with respect to the robot.
(z) Time-series for s01-s04 for less and more sync vertical movement with respect to the robot.
}
\label{fig:main}
\end{figure*}






\section{FUTURE WORK}

Collect data from more participants whom will perform complex movements in groups.
Take advantage of the Deep Neural Networks to automatically clasify the quality of the movements 
according to how well the participants are in sync with the movements.
%\end{document}  % This is where a 'short' article might terminate

%ACKNOWLEDGMENTS are optional
\section{Acknowledgments}
XX XX is supported by XXX. The support
is gratefully acknowledged
%
% The following two commands are all you need in the
% initial runs of your .tex file to
% produce the bibliography for the citations in your paper.
\bibliographystyle{abbrv}
\bibliography{literature}  % sigproc.bib is the name of the Bibliography in this case
% You must have a proper ".bib" file
%  and remember to run:
% latex bibtex latex latex
% to resolve all references
%
% ACM needs 'a single self-contained file'!
%


% %APPENDICES are optional
% %\balancecolumns
% \appendix
% %Appendix A
% \section{Headings in Appendices}
% The rules about hierarchical headings discussed above for
% the body of the article are different in the appendices.
% In the \textbf{appendix} environment, the command
% \textbf{section} is used to
% indicate the start of each Appendix, with alphabetic order
% designation (i.e. the first is A, the second B, etc.) and
% a title (if you include one).  So, if you need
% hierarchical structure
% \textit{within} an Appendix, start with \textbf{subsection} as the
% highest level. Here is an outline of the body of this
% document in Appendix-appropriate form:
% \subsection{Introduction}
% \subsection{The Body of the Paper}
% \subsubsection{Type Changes and  Special Characters}
% \subsubsection{Math Equations}
% \paragraph{Inline (In-text) Equations}
% \paragraph{Display Equations}
% \subsubsection{Citations}
% \subsubsection{Tables}
% \subsubsection{Figures}
% \subsubsection{Theorem-like Constructs}
% \subsubsection*{A Caveat for the \TeX\ Expert}
% \subsection{Conclusions}
% \subsection{Acknowledgments}
% \subsection{Additional Authors}
% This section is inserted by \LaTeX; you do not insert it.
% You just add the names and information in the
% \texttt{{\char'134}additionalauthors} command at the start
% of the document.
% \subsection{References}
% Generated by bibtex from your ~.bib file.  Run latex,
% then bibtex, then latex twice (to resolve references)
% to create the ~.bbl file.  Insert that ~.bbl file into
% the .tex source file and comment out
% the command \texttt{{\char'134}thebibliography}.
% % This next section command marks the start of
% % Appendix B, and does not continue the present hierarchy
% \section{More Help for the Hardy}
% The sig-alternate.cls file itself is chock-full of succinct
% and helpful comments.  If you consider yourself a moderately
% experienced to expert user of \LaTeX, you may find reading
% it useful but please remember not to change it.
% %\balancecolumns % GM June 2007
% % That's all folks!



\end{document}
